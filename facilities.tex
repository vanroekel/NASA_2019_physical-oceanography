%\begingroup
%\setlength{\parindent}{0pt}
%\setlength{\parskip}{24pt plus 24pt}

\subsection*{Los Alamos National Laboratory}

Researchers at LANL and their collaborators have access to substantial
state-of-the-art Institutional Computing capabilities as part of institutional
support and ongoing funding by DOE Climate Modeling programs. 
These resources
come in the form of production systems that are part of LANL's
Institutional Computing resources and additional set of testbed
systems devoted entirely to supporting research and development
activities.
Current funding permits an allocation between 5-10\% of the resources below
with additional resources available through an annual proposal-driven
allocation. PI Van Roekel already possesses allocations on these systems and has a strong track record of acquiring resources through the open proposal call.


The current available institutional computing systems include: 
% from https://hpc.lanl.gov/summary_table
\begin{center}
  \begin{tabular}{|p{0.75in}|p{0.7in}|p{0.6in}|p{0.5in}|p{0.45in}|p{0.75in}|l|}
  \hline 
  \textbf{System} & \textbf{Processor Type} & \textbf{Memory (GB)} & \textbf{Cores / node} & \textbf{Node Count} & \textbf{Network} & \textbf{TFLOPS}\\
  \hline
  \hline 
  Grizzly & Intel Xeon Broadwell & 128 & 36  & 1,490 & OmniPath & 1,806  \\
  \hline
  Badger    & Intel Xeon Broadwell & 128 & 36 & 660 & Omnipath &  798 \\
  \hline
  Kodiak     & Intel Broadwell / NVIDIA Tesla P100 GPGPUs & 256 & 36 & 133 & Infiniband Enhanced Data Rate   & 1850 \\
  \hline
  \end{tabular}
\end{center}

All of these clusters are attached to a 10PB Lustre high performance storage
system and expandable archival storage that currently has a capacity of several
hundred PB. A small high-memory visualization cluster is available for
analysis.

In addition to these production systems, Laboratory researchers host a number
of small clusters and single-node systems that are used to explore Nvidia GPUs
and Intel many-core architectures. 
For example, one collection of testbed systems can be accessed through
a common front-end dubbed Darwin.  This 128 node cluster is used
to explore higher core count designs and accelerator-based computing.
The majority of Darwin's nodes contain 48 AMD Opteron cores and either
one or two GPUs.  Half of the nodes contain dual NVIDIA Fermi-class
GPUs and the other half a single AMD GPU. The nodes are interconnected
with dual Infiniband QDR interfaces.  In addition, each node contains
both a solid state and a mechanical disk drive.  The system contains
additional nodes that explore new processors and systems from Intel,
AMD, NVIDIA, and ARM. 

%\endgroup



