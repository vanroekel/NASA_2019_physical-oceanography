\begin{center}
{\Large{\bf TITLE:}}\\*[3mm]
{\bf Proposal Title} \\*[3mm]

PI Names \\
More PI Names

\end{center}


\textbf{ABSTRACT:}
Coastal upwelling is an essential process to the coastal ecosystem and is also a critical component of the marine boundary layer.  If coastal upwelling weakens, perhaps in response to greenhouse gas forcing (e.g. Balkan 1990. Snyder et al. 2003), the stratocumulus coverage could weaken, further exacerbating global warming.  Therefore, understanding the physical drivers of coastal upwelling and how they might change as the climate changes is critical.  Previous work (e.g. Small et al. 2012) suggests that models must have fine resolution in the atmosphere and ocean components to accurately simulate upwelling.  Given the computational expense of global high resolution simulations in ocean and atmosphere models, most studies of coastal upwelling (e.g., Chhak and DiLorenzo 2007 and Snyder et al. 2003) utilize regional ocean model simulations alone, ignoring potentially important teleconnections through the atmosphere and ocean to the tropical ocean (e.g. El Nino Southern Oscillation).  In this proposal the science questions are focused on understanding how tropical phenomena (El Nino) influence coastal upwelling through oceanic pathways.  This could be equatorial Kelvin waves propagating along the equatorial Pacific and then northward as coastally trapped waves (CTW) along the western U.S. coast, influencing the upwelling and the coastal water properties. And also the California Undercurrent (CUC), that has been reported to influence the water properties (Stone et al. 2017 and Hickey et al. 2016).  In particular we seek to quantify relationships between ENSO and its influence on upwelled waters along the US Western coast.   We also seek to understand how the linkage between ENSO and coastal upwelling change when we consider the two distinct "flavors" of El Nino: the traditional Eastern Pacific version, and the Central Pacific or dateline ENSO.  We will also explore how the linkage changes in different phases of large scale climate modes (e.g. the Pacific Decadal Oscillation).  Finally we will assess how model uncertainty changes with an improved understanding of the oceanic pathways of tropical Pacific influence on coastal upwelling.

We will conduct this work in two phases.  In the first, we will leverage a number of NASA products to understand coastal upwelling, CUC and, CTW most critically will be the GHRSST datasets, sea surface height (e.g. TOPEX/POSEIDON or JASON2).  We will also explore how the linkages are altered with resolution by utilizing the ECCO1 and ECCO2 day assimilation products.  The latter product fully resolves mesoscale eddies, but is only valid for the past 20 years.  To understand how upwelling and model uncertainty responds to climate change we will require use of a earth system model.  For this work, we will leverage the unique variable resolution capability to accurately simulate coastal upwelling without requiring global high resolution.  We also will explore pushing the boundaries of resolution toward the submesoscales (~1km).  We will conduct these simulations in a staged manner (using data atmosphere and active atmosphere simulations) to assess the impact of resolution and coupling on the simulation of coastal upwelling.
